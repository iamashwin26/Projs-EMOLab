\documentclass[11pt]{article}
\usepackage{siunitx}
\marginparwidth 0.5in 
\oddsidemargin 0.25in 
\evensidemargin 0.25in 
\marginparsep 0.25in
\topmargin 0.25in 
\textwidth 6in \textheight 8 in
\usepackage{multirow}
\usepackage{tabularx}
\usepackage{longtable}
\usepackage{booktabs}
\usepackage{amssymb}
\usepackage{pgfplots}
\usepackage{pgfplotstable}
\usepackage{float}
\usepackage{empheq}
\usepackage{mathtools}
\usepackage{longtable}

\pgfplotstableset{% global config, for example in the preamble
  every head row/.style={before row=\toprule,after row=\midrule},
  every last row/.style={after row=\bottomrule},
  fixed,precision=2,
}


\begin{document}

\author{Ashwin Kumar K - 2017B5PS1034G}
\title{PHYF214 PHYSICS LAB REPORT SEM1 2018-2019 \\ Lab 5 Group 7:  Spectroscopy using a Diffraction Grating [DG]
}
\date{6 \textsuperscript{th} September, 2018}
\maketitle

\section{Experimental Tasks}
\begin{enumerate}
    \item To calibrate the grating spectrometer using the known source (Hg source) of light and to calculate the grating constant.
    \item Using the same grating, to calculate the wavelength of sodium doublet lines.
\end{enumerate}
\section{Apparatus}
Spectrometer, transmission diffraction grating, sodium vapour lamp, mercury vapour lamp,
power supply for spectral lamps, magnifying glass.

\section{Theory}
Light from a mercury source is incident normally on a diffraction grating mounted on a spectrometer after calibration.
The diffraction angle of the diffracted light is measured for each spectral line of the
Hg-source.This enables us to calculate a constant d using the formula  
\begin{equation}
    dsin\theta=\lambda
\end{equation}

d is related to the no of slits in the grating.\\
Likewise for sodium source, the diffraction angle and angular separation $\Delta \theta $ of the
sodium doublet is measured. Then using equation (1) we can back calculate the wavelength $\lambda$ of the two lines.




\section{Observations and Analysis}
\subsection{Calibration of the Diffraction Grating}
Least count of the scale = $\frac{1}{60}^{\circ}$.\\
Table 1: Data gathered while making the slit perpendicular to collimator .

\begin{table}[h]
\begin{tabular}{|l|l|l|l|}
\hline
Angle for step 4(a )& Angle for step (4b) & Angle for step (4c) & Angle for step (4d) \\
\hline
348.33$^{\circ}$            & 78.33$^{\circ}$             & 124$^{\circ}$               & 79$^{\circ}$             \\
\hline
\end{tabular}
\end{table}
Table 2: Data gathered during calibration with mercury lamp.\\
\begin{table}[h]
\begin{tabular}{|p{1.5cm}|p{2cm}|p{2cm}|p{2cm}|l|l|l|}
\hline
Colour     & wavelength (in m) & LHS reading (in$^{\circ}$ )         & RHS reading (in$^{\circ}$ )        & Angle (in$^{\circ}$ )       & $sin\theta$        & d= $\frac{\lambda}{sine\theta}$ \\
\hline
Violet 1   & 4.05E-007  & 2.03 & 337.6       & 12.216 & 0.211 & 1.91E-006   \\
Violet 2   & 4.08E-007  & 2.93 & 336.67 & 13.133 & 0.227 & 1.79E-006   \\
Blue-green & 4.92E-007  & 3.53 & 335.2       & 14.166 & 0.244 & 2.01E-006   \\
Green      & 5.46E-007  & 5.16 & 333.53 & 15.817 & 0.272 & 2.00E-006   \\
Yellow 1   & 5.77E-007  & 6.16 & 332.58 & 16.797 & 0.288 & 2.00E-006   \\
Yellow 2   & 5.79E-007  & 6.3         & 333.5       & 16.4        & 0.2823414587 & 2.05E-006  \\ \hline
\end{tabular}
\end{table}

\begin{figure}[H]
\caption{Graph of $\lambda$ vs $sin\theta$ for the calibration part}\par\medskip
\begin{tikzpicture}
\begin{axis}[
    axis lines = left,
    xlabel={$sin\theta$ $\Rightarrow$},
    ylabel=$\lambda$(in $m$) $\Rightarrow$,
    ]
  \addplot [
    color=red,
] table [x=sin@,y={wavelength}] {Table2.txt};

\end{axis}
\end{tikzpicture}
\end{figure}
slope of the graph $d = 2.03E-06$ m.\\

\textbf{\emph{Analysis:}}\\
Error in value of d can be given by the standard deviation $\sigma = 9.3 E -8$ m 
\subsection{Determination of wavelength of sodium doublet lines}

\pagebreak
Table 3: Data gathered with sodium lamp.
\begin{table}[h]
\begin{tabular}{|l|l|l|l|l|l|}
\hline
Colour   & LHS reading (in$^{\circ}$ )        & RHS reading (in$^{\circ}$ )        & $\theta$ (in $^\circ$)           & $sin\theta$         & $\lambda$ (in m)                    \\ \hline
Yellow 1 & 5.8167 & 333.3833 & 16.2167 & 0.279 & 5.48E-007   \\
Yellow 2 & 6.533 & 332.33 & 17.1        & 0.294 & 5.77E-007  \\
\hline
\end{tabular}
\end{table}


\textbf{\emph{Analysis:}}\\
To calculate the error in $\lambda$ we make use of the equation 
\begin{equation}
    \frac{\Delta \lambda}{\lambda} = \frac{\Delta d}{d} + \frac{\Delta \theta}{\theta}\\
    
\end{equation}
$\therefore $ we get $ \Delta \lambda$ for yellow 1 as 1.74E-7 and that for yellow 2 as 1.92E-7. 
\section{Precautions}
\begin{itemize}

    \item  The directions of rotation of the telescope micrometer screw should be maintained the
same. Otherwise the play in the micrometer spindle will lead to backlash errors.
    \item  The experiment should be performed in a dark room.
    \item  The spectral lamps should attain their full illuminating power after being warmed up for
about 5 minutes, so the observations should be taken after 5 minutes.
    \item One of the essential precautions for the success of this experiment is to set the grating
normal to the incident rays. Small variation in the angle of incidence causes large error
in the angle of diffraction. If the exact normality is not achieved, one finds that the angles
of diffraction measured on the left and on the right are not exactly equal.
\end{itemize}

\section{Conclusions and Results}
By following the standard procedure for calibrating the diffraction grating and measurement of sodium divergence angle , we were able to find two lines in the sodium emission spectra namely yellow 1 and yellow 2 lines peaking at $(5.48 \pm 1.74)E -7 $ m and $(5.77 \pm 1.92)E -7 $m respectively.



\end{document}
